\documentclass[11pt,article]{article}
\usepackage[utf8]{inputenc}
\usepackage[T1]{fontenc} % caractères accentués en entrée, dans emacs
\usepackage[french]{babel}
\FrenchFootnotes
\selectlanguage{french}
\usepackage{a4wide} % possibilité d'utiliser toute la page a4
% selon GUT#33, avril 2007, page 13, empagement
% largeur des textes (ou justification) = 15cm
% hauteur du rectangle d'empagement = 23cm
% blanc de couture = 2/5 (21-15) = 2.4 = inner = right
% blanc de grand fond = 3/5 (21-15) = outer = left
% blanc de tête = 2/5 (29,7-23) = top
% blanc de pied = 3/5 (29,7-23) = bottom
%\usepackage[a4paper,twoside=true,right=2.4cm,left=3.6cm,top=2.68cm,bottom=4.02cm]{geometry}
% selon CFSE 2006
% - largeur des textes (ou justification) : 16cm (2cm de marge, et 1cm
%   de reliure) ;
% - hauteur des textes, y compris les notes : 23cm (2,5cm de marge
%   haute et 2cm de marge basse) ; 1ère page de : 36pts
%   d'espacement avant le titre ;
\oddsidemargin   -4mm           % 3cm a gauche des impaires
\evensidemargin   4mm           % 2cm a gauche des paires
\topmargin       -18mm          % 2.5cm en haut
\headheight       13mm          % taille de l'entete (lignes)
\headsep          24pt          % espace entre entete et texte
\footskip         30pt          % espace entre pied de page et texte
\textheight      230mm          % longeur du texte
\textwidth       160mm          % largeur du texte
\parskip 1pt                    % pas de sauts entre paragraphes
%\parindent 0pt                  % largeur de l'indentation
\usepackage{graphicx} % figure postcript avec latex,
		      % figure png avec pdflatex, au lieu d'utiliser epsfig
\usepackage[usenames,dvipsnames,table]{xcolor}
\usepackage[export]{adjustbox}
\usepackage{paralist}
\usepackage{ifthen}
\usepackage{amssymb}
\usepackage{amsfonts}
\usepackage{amsmath}
\usepackage{eurosym}
\usepackage{textcomp}
\usepackage{listings}
\lstset{language=Java,numbers=left,numberstyle=\tiny,stepnumber=4,numbersep=5pt,xleftmargin=5pt}

\usepackage{alltt}
\usepackage{longtable}

% adjust word spacing less strictly
% as result, some spaces between words may be a bit too large,
% but long words will be placed properly.
\sloppy

\newcommand{\cmt}[1]{\texttt{<}\textbf{--~#1~--}\texttt{>}}

\usepackage{lineno}
\usepackage{xspace}

\setlength{\marginparwidth}{1cm}
\setlength{\marginparsep}{10pt}
\reversemarginpar
\newcounter{usecasehaute}
\newcommand{\haute}{Haute}
\newcommand{\moyenne}{Moyenne}
\newcommand{\basse}{basse}
\newcommand{\usecase}[4]{\item \marginpar{\vspace{5pt}\ifthenelse{\equal{#1}{Haute}}{\centering\textsc{#1}\stepcounter{usecasehaute}\newline n$^{\circ}$ \theusecasehaute}{\ifthenelse{\equal{#1}{Moyenne}}{#1}{\small #1}}} #2 \begin{itemize}\item précondition~: #3 \item postcondition~: #4\end{itemize}}
\newcommand{\priorityusecase}[2]{\item \marginpar{\vspace{5pt}\ifthenelse{\equal{#1}{Haute}}{\centering\textsc{#1}\stepcounter{usecasehaute}\newline n$^{\circ}$ \theusecasehaute}{\ifthenelse{\equal{#1}{Moyenne}}{#1}{\small #1}}} #2}
\newcommand{\casusecase}[4]{\usecase{#1}{#2}{#3}{#4}}

\newcommand{\nullvalue}{\textsf{null}\xspace}
\newcommand{\emptyvalue}{\ensuremath\mathrm{vide}}
\newcommand{\invariant}{\ensuremath\mathrm{invariant}}

\begin{document}
\title{Projet CSC4102: Gestion des clefs dans un hôtel}
\author{Huang ShiHui et Mabileau Paul}
\date{Année 2019--2020~---~\today}
\maketitle

\newpage

\tableofcontents

\newpage

\section{Spécification}

\subsection{Diagrammes de cas d'utilisation}

\begin{figure}[h!]
\begin{center}
\includegraphics[scale=0.5]{DiagrammesDeCasDUtilisation/gestionclefshotel_uml_diag_cas_utilisation}
\caption{Diagramme de cas d'utilisation}
\end{center}
\label{umlet_diag_cas_utilisation}
\end{figure}

\newpage

\subsection{Priorités, préconditions et postconditions des cas d'utilisation}

\begin{compactitem}
\usecase{\haute}{Créer une chambre}
        %% précondition
        {identifiant/code de la chambre bien formé (non \nullvalue et non
          vide)
          \\$\land$ chambre avec ce code inexistante \\$\land$ graine pour la génération des clefs bien formée (non
          \nullvalue et non vide)}
        %% postcondition
        {chambre avec cet identifiant existante}

\smallskip

\usecase{\haute}{Créer un badge d'access}
        %% précondition
         {}
        %% postcondition
         {\\badge vierge}

\smallskip

\usecase{\haute}{Créer un client}
               %% précondition
                    {\\nom prénom et identifiant du client bien formées (non \nullvalue et non vide)\\
                    $\land$ client non existant dans le système}
                %% postcondition
                    {\\client enregistré dans le système}

\smallskip
\usecase{\haute}{(Re)Initialiser la serrure d'une chambre }
                %% précondition
                    {\\identifiant de la serrure bien formé (non \nullvalue et non vide)\\
                    $\land$ graine et sel pour la génération des clefs bien formée (non \nullvalue et non vide)}
                %% postcondition
                    {\\serrure initialisé}


\smallskip

            \usecase{\haute}{Enregistrer l'occupation d'une chambre par un client}
                %% précondition
                    {\\nom et prénom du client bien formés (non \nullvalue et non vide)\\
                        $\land$ client existe\\
                        $\land$ client occupe aucune chambre\\
                        $\land$ identifiant/code de la chambre bien formé (non \nullvalue et non vide)\\
                        $\land$ chambre avec ce code existante\\
                        $\land$ chambre non occupée\\
                     	$\land$ badge d'accèss disponible\\
                %% postcondition
                    {\\badge d'accès initialisé\\
                        $\land$ paire de clés du badge d'accès bien formées (non \nullvalue et non vide)\\
                        $\land$ +1 sur nombre de chambre occupée en cours du client\\
                    $\land$ chambre occupée}

            \smallskip

           \usecase{\haute}{Libérer une chambre}
                %% précondition
                    {\\
                        $\land$ client existe\\
                        $\land$ client occupe une chambre\\
                        $\land$ identifiant/code de la chambre bien formé (non \nullvalue et non vide)\\
                        $\land$ chambre occupée\\
                %% postcondition
                    {\\
                        $\land$ -1 sur nombre de chambre occupée en cous du client\\
                    $\land$ chambre non occupée}

            \smallskip


\priorityusecase{\basse}{Retirer une chambre}

\smallskip

\priorityusecase{\moyenne}{Lister les chambres}
\end{compactitem}

\newpage

\section{Préparation des tests de validation}

\subsection{Tables de décision des tests de validation}

La fiche programme du module CSC4102 ne permettant pas de développer
des tests de validation couvrant l'ensemble des cas d'utilisation de
l'application, les cas d'utilisation choisis sont de priorité
\textsc{Haute}.

\begin{table}[htbp!]
\begin{tabular}{|p{0.6\linewidth}|c|c|c|c|}
\hline
Numéro de test
&1&2&3&4\\
\hline
\hline
Identifiant/code de la chambre bien formé (non \nullvalue et non vide)
&F&T&T&T\\
\hline
Graine pour la génération des clefs bien formée ($\neq$ \nullvalue $\land$ $\neq$ vide)
& &F&T&T\\
\hline
Chambre inexistante avec ce code
& & &F&T\\
\hline
\hline
Création acceptée
&F&F&F&T\\
\hline
\hline
Nombre de jeux de test
&2&2&1&1\\
\hline
\end{tabular}
\caption{Cas d'utilisation <<~créer une chambre~>>}
\end{table}
        \begin{table}[htbp!]
            \begin{tabular}{|p{0.6\linewidth}|c|c|c|c|c|c|c|c|}
                \hline
                Numéro de test
                    &1&2&3&4&5&6&7&8\\
                \hline
                \hline
                Nom et prénom du client bien formés (non \nullvalue et non vide)
                    &F&T&T& & & & &T\\
                \hline
                Client existe
                    & &F&T& & & & &T\\
                \hline
                Client occupe aucune chambre
                    & & &F& & & & &T\\
                \hline
                \hline
                Identifiant/code de la chambre bien formé ($\neq$ \nullvalue $\land$ $\neq$ vide)
                    & & & &F&T&T&T&T\\
                \hline
                Chambre avec ce code existante
                    & & & & &F&T&T&T\\
                \hline
                Chambre non occupée
                    & & & & & &F&T&T\\
                \hline
                Dernière paire de clefs de la chambre bien formé ($\neq$ \nullvalue $\land$ $\neq$ vide)
                    & & & & & & &F&T\\
                \hline
                \hline
                Enregistrement accepté
                    &F&F&F&F&F&F&F&T\\
                \hline
                \hline
                Nombre de jeux de test
                                    &2&1&1&2&1&1&2&2 \\
                \hline
            \end{tabular}
            \caption{Cas d'utilisation <<~enregistrer l'occupation d'une chambre par un client~>>}
        \end{table}

        \begin{table}[htbp!]
            \begin{tabular}{|p{0.6\linewidth}|c|c|c|c|c|c|c|c|}
                \hline
                Numéro de test
                    &1&2&3&4&5&6&7&8\\
                \hline
                \hline
                Nom et prénom du client bien formés (non \nullvalue et non vide)
                    &F&T&T& & & & &T\\
                \hline
                Client existe
                    & &F&T& & & & &T\\
                \hline
                Client occupe une chambre
                    & & &F& & & & &T\\
                \hline
                \hline
                Identifiant/code de la chambre bien formé ($\neq$ \nullvalue $\land$ $\neq$ vide)
                    & & & &F&T&T&T&T\\
                \hline
                Chambre avec ce code existante
                    & & & & &F&T&T&T\\
                \hline
                Chambre occupée
                    & & & & & &F&T&T\\
                \hline
                Paire de clés du badge d'accès bien formées ($\neq$ \nullvalue $\land$ $\neq$ vide)
                    & & & & & & &F&T\\
                \hline
                \hline                Libération acceptée
                    &F&F&F&F&F&F&F&T\\
                \hline
                \hline
                Nombre de jeux de test
                    &2&1&1&2&1&1&2&2 \\
                \hline
            \end{tabular}
            \caption{Cas d'utilisation <<~libérer une chambre~>>}
        \end{table}
\newpage

\section{Conception}

\subsection{Liste des classes}

À la suite d'un parcours des diagrammes de cas d'utilisation et d'une
relecture de l'étude de cas, voici la liste de classes avec quelques
attributs:
\begin{compactitem}
\item \textsf{GestionClefsHotel} (la façade)
\item \textsf{Chambre}~---~identifiant, graine, sel
\item \textsf{Client}~---~identifiant, nom, prénom (ces deux derniers
  sont ajoutés ici mais ne sont pas essentiels au fonctionnement du
  système \textsf{GestionClefsHotel})
\item \textsf{Badge}~---~identifiant
\item \textsf{PaireClefs}~---~clef1, clef2
\item \textsf{Util} (classe utilitaire déjà programmée)~---~'attribut
  de classe \textsf{TAILLE\_CLEF}, méthodes de classe
  \textsf{genererUneNouvelleClef} et \textsf{clefToString})
\end{compactitem}
\newpage

\subsection{Diagramme de classes}

\begin{figure}[h!]
  \includegraphics[width=1.3\textwidth,center]{DiagrammesDeClasses/gestionclefshotel_uml_diag_classes}
  \caption{Diagramme de classes}
  \label{umlet_diag_classes}
\end{figure}

\newpage

\subsection{Diagrammes de séquence}

\begin{figure}[h!]
  \includegraphics[width=1.3\textwidth,center]{DiagrammesDeSequence/DSUC1}
  \caption{Diagramme de séquence DSUC1 : <<Créer une chambre>>}
  \label{umlet_diag_seq1}
\end{figure}

\newpage

\begin{figure}[h!]
  \includegraphics[width=1.3\textwidth,center]{DiagrammesDeSequence/DSUC2}
  \caption{Diagramme de séquence DSUC2 : <<Enregistrer l'occupation d'une chambre par un client>>}
  \label{umlet_diag_seq2}
\end{figure}

\newpage

\begin{figure}[h!]
  \includegraphics[width=1.3\textwidth,center]{DiagrammesDeSequence/DSUC3}
  \caption{Diagramme de séquence DSUC3 : <<Libérer une chambre>>}
  \label{umlet_diag_seq3}
\end{figure}

\newpage

\section{Fiche des classes}

{\color{red}\textbf{La section est à compléter avec les fiches de vos
    classes les plus importantes. La première fiche, celle de la
    façade, est aussi à compléter.}}

\subsection{Classe \textsf{GestionClefsHotel}}

\begin{center}
\begin{longtable}{|p{15cm}|}
\hline
\multicolumn{1}{|c|}{{\Large \textsf{GestionClefsHotel}}} \\
\hline
%\cmt{attributs}\\
\cmt{attributs <<~association~>>}\\
$-$ chambres : collection de @Chambre \\
\hline
\cmt{constructeur} \\
$+$ GestionClefsHotel()\\
%$+$ destructeur()\\
$+$ invariant():booléen\\
\cmt{operations <<~cas d'utilisation~>>} \\
$+$ créerChambre(String code, String graine) \\
%\cmt{opérations de recherche} \\
\hline
\end{longtable}%)
\end{center}

\newpage

\section{Diagrammes de machine à états et invariants}

{\color{red}\textbf{La section est à compléter avec les diagrammes de
    machine à états et les invariants de vos classes les plus
    importantes.}}

\newpage

\section{Préparation des tests unitaires}

{\color{red}\textbf{La section est à compléter avec les tables de
    décision de certaines méthodes des classes les plus importantes.}}


\end{document}
